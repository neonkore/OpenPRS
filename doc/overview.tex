%%%%%%%%%%%%%%%%%%%%%%%%%%%% -*- Mode: latex -*- %%%%%%%%%%%%%%%%%%%%%%%%%%
%%
%% Copyright (c) 1991-2003 Francois Felix Ingrand.
%% All rights reserved.
%%
%% Redistribution and use in source and binary forms, with or without
%% modification, are permitted provided that the following conditions
%% are met:
%%
%%    - Redistributions of source code must retain the above copyright
%%      notice, this list of conditions and the following disclaimer.
%%    - Redistributions in binary form must reproduce the above
%%      copyright notice, this list of conditions and the following
%%      disclaimer in the documentation and/or other materials provided
%%      with the distribution.
%%
%% THIS SOFTWARE IS PROVIDED BY THE COPYRIGHT HOLDERS AND CONTRIBUTORS
%% "AS IS" AND ANY EXPRESS OR IMPLIED WARRANTIES, INCLUDING, BUT NOT
%% LIMITED TO, THE IMPLIED WARRANTIES OF MERCHANTABILITY AND FITNESS
%% FOR A PARTICULAR PURPOSE ARE DISCLAIMED. IN NO EVENT SHALL THE
%% COPYRIGHT HOLDERS OR CONTRIBUTORS BE LIABLE FOR ANY DIRECT, INDIRECT,
%% INCIDENTAL, SPECIAL, EXEMPLARY, OR CONSEQUENTIAL DAMAGES (INCLUDING,
%% BUT NOT LIMITED TO, PROCUREMENT OF SUBSTITUTE GOODS OR SERVICES;
%% LOSS OF USE, DATA, OR PROFITS; OR BUSINESS INTERRUPTION) HOWEVER
%% CAUSED AND ON ANY THEORY OF LIABILITY, WHETHER IN CONTRACT, STRICT
%% LIABILITY, OR TORT (INCLUDING NEGLIGENCE OR OTHERWISE) ARISING IN
%% ANY WAY OUT OF THE USE OF THIS SOFTWARE, EVEN IF ADVISED OF THE
%% POSSIBILITY OF SUCH DAMAGE.
%%
%% $Id$
%%
%%%%%%%%%%%%%%%%%%%%%%%%%%%%%%%%%%%%%%%%%%%%%%%%%%%%%%%%%%%%%%%%%%%%%%%%%%%%%%%

\part{Overview}
\node{Overview}



\chapter*{Overview of the \COPRSDE{}}
\node{Overview of the OPRS Development Environment}
\cpindex{Overview of the OPRS Development Environment}
\cpindex{OPRS Development Environment Overview}


The \COPRSDE{} is a set of programs designed to help users create applications using 
Procedural Reasoning \cite{Ingrand-Rao-92}. The concept of
Procedural Reasoning appeared a few years ago at SRI International in Menlo Park,
California. To date, it has been the subject of many research projects and numerous 
publications have been written on this new programming
paradigm (see Bibliography \pageref{bibliography}).

\COPRSDE{} is, to our knowledge, the first complete Procedural Reasoning
development environment written in C and available under Unix. Its graphic
interface is based on MIT's X Window version 11 ( X11), and
Open Software Foundation's Motif widget set (Motif).
It is the first implementation of \OPRS{} intended as a
complete software product and environment, unlike other previous versions
which were research prototypes usually written in different Lisp dialects.



\section{What is Procedural Reasoning?}
\node{What is Procedural Reasoning?}
\cpindex{What is Procedural Reasoning?}
\cpindex{Procedural Reasoning}

Procedural Reasoning is a set of tools and methods representing and executing
plans and procedures. These plans or procedures are conditional sequences of
actions which can be run to achieve given goals or to react in particular
situations. Procedural Reasoning differs from other commonly used knowledge
representation (rules, frames, etc.)

To a degree, procedural representation is a trade-off between purely
declarative representations and strictly imperative representations. Declarative 
representations suffer from a lack of control on the execution of their rules, imperative
suffer from limited modularity. Procedural Reasoning is particularly well suited
for problems where implicit or explicit knowledge is already formalized as
procedures or plans. The control information (i.e. the sequence of actions and
tests) embedded in these procedures or plans is actually preserved.

\section{Overall Description of \COPRS{}}
\node{Overall Description of OPRS}
\cpindex{Overall Description of OPRS}

The C Procedural Reasoning System (\COPRS{}) is a generic architecture for
representing and reasoning about actions and procedures in a dynamic domain. It
has been applied to various tasks with real-time demands, including malfunction
monitoring for different subsystems of NASA's space shuttle
\cite{Georgeff-Ingrand90-a}, the diagnosis, monitoring and control of
telecommunications networks \cite{Wesley91}, the control of  mobile robots
\cite{Revillod92} and system control for a surveillance aircraft
\cite{Ingrand89}.

\figuregif[scale=1]{global-arch}{OPRS Global Architecture}{global-arch}

As shown in Figure \ref{global-arch}, the architecture of a \CPK{} consists of
(1) a database containing the system of current beliefs about the world; (2) a
library of plans (or procedures), called Knowledge Areas (OPs), that describe
particular sequences of actions and tests that may be performed to achieve
given goals or to react to certain situations; and (3) an intention graph,
consisting of a [partially] ordered set of all plans chosen for execution at
runtime. An interpreter (inference mechanism) manipulates these components,
selecting an appropriate plan (OP) based on system beliefs and goals, placing
those selected OPs on the intention structure, and finally executing them.

The \CPK{} interacts with its environment through its database, which
acquires new beliefs in response to changes in the environment, and through the
actions it performs as it carries out its intentions. Different instances
of \COPRS{}, running asynchronously, can be used in an application that requires
the cooperation of more than one subsystem.

In \COPRS{}, goals are descriptions of desired tasks or behaviors. In the
logic used by \COPRS{}, the goal to achieve a certain condition \code{C} is
written as \code{(! C)}; the test for a condition  is written as \code{(? C)};
to wait until the condition is true is written as \code{(\^{} C)}; to passively
maintain \code{C} is written as \code{(\# C)}; to actively maintain \code{C} is
written as \code{(\% C)}; to assert the condition \code{C} is written as
\code{(=> C)}; and to retract the condition \code{C} is written as \code{(~>
C)}. For example, the goal to close valve \code{v1} could be represented as
\code{(! (position v1 cl))}, and to test for it being closed as \code{(?
(position v1 cl))}.

Knowledge about how to accomplish given goals or to react to certain situations
is represented in \COPRS{} by declarative procedure specifications called
\dfn{Knowledge Areas} (OPs). Each OP consists of a \dfn{body}, which describes
the steps of the procedure, and an \dfn{invocation condition}, which specifies
under which situations the OP is useful. Together, the invocation condition and
body of a OP express a declarative fact about the results and utility of
performing certain sequences of actions under certain conditions
\cite{Georgeff-Lansky86-a}.

The set of OPs in a \COPRS{} application system not only consists of procedural
knowledge about a specific domain, but also includes \dfn{meta level} OPs ---
that is, information about the manipulation of the beliefs, goals, and
intentions of \COPRS{} itself. For example, typical meta level OPs encode various
methods for choosing among multiple applicable OPs. They determine how to achieve
a conjunction or disjunction of goals, and compute the amount of additional
reasoning that can be undertaken, given the real-time constraints of the
problem domain. In achieving this, meta level OPs make use of information
about OPs that are contained in the system database or in the property slots of
the OP structures.

\COPRS{} has several features that make it particularly powerful as a
situated reasoning system, including: (1) the semantics of its plan (procedure)
representation, which is important for verification and maintenance; (2) its
ability to construct and act upon partial (rather than complete) plans; (3) its
ability to pursue goal-directed tasks while at the same time be responsive
to changing patterns of events in bounded time; (4) its facilities for managing
multiple tasks in real-time; (5) its default mechanisms for handling stringent
real-time demands of its environment; and (6) its meta level (or reflective)
reasoning capabilities. Some of these features have been discussed in earlier
reports and papers
\cite{Georgeff-Ingrand89,Georgeff-Ingrand90-b,Georgeff-Ingrand90-a}.

\section{Example of Procedure/OP in \COPRS{}}
\node{Example Procedure/OP in OPRS}
\cpindex{Example Procedure/OP in \COPRS{}}

The procedure or OP which is presented on Figure \ref{fig-proc1-fig}
belongs to a library of procedures in the control and supervision of a tank
truck filling station. Its goal is to appropriately close or open the filling
valve. To control the proper execution, it monitors two position sensors placed
on the valve. It generates an emergency stop of the
station in case of malfunction.

\figuregif[scale=0.5]{move-valve-op-exp}{Example of a OP}{fig-proc1-fig}

The points shown on the figure \ref{fig-proc1-fig} are now presented and
developed:

\begin{enumerate}

\item The \dfn{name} of the procedure allows the user to designate it in the
different selection menus of the system.

\item The \dfn{invocation part} specifies which goals or which events may make it
applicable. This particular procedure is applicable whenever the system has the
goal to put the valve in position \code{\$X} (\code{\$X} is a variable which can take two
values: ``open'' or ``closed'', which will be known at run-time).

\item The \dfn{context part} further specifies the conditions of applicability of
the procedure. In this case, it will determine the acceptable response delays
on the valve position sensors.

\item The \dfn{effects part} specifies the facts you want to add or retract
from the database upon successful execution of the procedure. Here, if the
procedure successfully executes, it will conclude the new position of the valve
in the database.

\item The \dfn{documentation} field is self explanatory.

  Then, there is the ``procedural'' part which specifies the sequence of tests
  and actions to evaluate when the procedure is executed.

\item The ``\code{START}'' node is the starting point of the procedure. To
successfully execute a procedure, one must satisfy all the actions and
conditions which lead to an ``\code{END}'' node. This is done by jumping from node to
node, while satisfying the condition which label the edge connecting this two
nodes. For the ``Move Valve'' procedure, the first action to be done is to put
the ``switch'' in the proper position: \code{\$X}. There is only one node
``\code{START}'' in a procedure.

\item There can be more than one outgoing edge from a specific node. In
this case, the system will try to satisfy one condition after another. As soon as a condition 
is satisfied, we can make the transition to node at the head of the
edge (the node at which the edge points).

\item Execution proceeds from node to node until it reaches an ``\code{END}'' node.
When one ``\code{END}'' node is reached, the execution of the procedure is
considered successful. If it was goal invoked, then this goal is
considered  achieved. In our case, the valve will  indeed be in
position \code{\$X}. If no ``\code{END}'' node can be reached, then the execution is
considered to be failed, and the goal to be achieved remains to
be satisfied.

\end{enumerate}

In a typical \COPRS{} application, one defines a library of such procedures and
OPs. These procedures are then loaded in a \CPK{} which will execute them
whenever they are applicable, i.e., when their invocation and context parts are
true.

\section{The \COPRSDE{}}
\node{The OPRS Development Environment}
\cpindex{The \COPRSDE{}}

The \COPRS{} Development Environment is designed to allow the user to implement
applications in control and supervision of complex systems, automatic execution
of predefined procedures, etc. Thanks to its modular architecture, it is easy
to integrate an application developed with \COPRS{} in already existing systems.

\figuregif[]{oprsde}{\COPRSDE{}}{fig-oprsde}

The \COPRS{} Development Environment is composed of different programs and
modules:

\begin{itemize}

\item A \OPE{}, which enables the user to create, edit and modify its
application procedures.

\item A \OPRSS{}, which enables the user to asynchronously manages a
number of \CPK{}s and \XPK{}s.

\item A \MP{} , which is the module which enables an application and external
modules to communicate with the different kernels of the \COPRS{} application.

\item Some \CPK{}s and \XPK{}s, which execute the procedures of your
application.

\end{itemize}

\typeout{This section refers to a licensing unit and licensing mechanism.  Probably obsolete.}

A \COPRS{} application is organized around a ``\MP{}'' module and a
``\OPRSS{}'' module, which constitute the licensing unit of the \COPRSDE{}.
However, one can run as many \CPK{}s or \XPK{}s as required by the application on any 
machine of the network.

\section{The \COPRSAE{}}
\node{The OPRS Application Environment}
\cpindex{The \COPRSAE{}}

The \COPRSAE{}  is designed to run \COPRS{} applications developed under the
\COPRSDE{}. It allows the user to execute the procedures exactly as they are
executed in the \COPRSDE{}. However, it does not allow the user to modify or
edit the existing procedures. This environment is particularly well suited for
a site using a \COPRS{} application developed by a third party.

According to the needs of your application, the \COPRSAE{} exists in two versions:

A graphic version, under X11/Motif, which enables the user to follow the
graphic execution of the procedures and the evolution of the task graph in the
\XPK{}s (Figure \ref{fig-oprsgae}).

\figuregif[]{oprsgae}{\COPRSAE{} (graphical version)}{fig-oprsgae}

An ASCII version which enables the user to execute the procedures in a standard
Unix environment (Figure \ref{fig-oprsaae}). This version, functionally
identical to the previous one, does not allow the user to follow the graphical
execution of procedures.

\figuregif[]{oprsaae}{\COPRSAE{} (ASCII version)}{fig-oprsaae}

Whatever version is chosen, the licensing mechanism stays the same; the
``Message Passer'' and the ``OPRS Server'' are the central and unique modules
around which as many as required by the application, \CPK{}s or \XPK{}s are
run.

\section{The Structure of this Manual}
\node{The Structure of this Manual}
\cpindex{The Structure of this Manual}

This manual is available in hard copy or as an on-line help function that you can access 
while using the
\COPRSDE{}.
The book is a standard manual, with parts, chapters, sections, appendices,
etc\dots{} The on-line help is structured in a similar way with each section
accessible through a series of menus or hypertext links.
Additionally, the appropriate section of the on-line manual is
directly accessed when you request help from the \COPRSDE{}. For example,
most dialog boxes in the Motif interface have a \code{HELP} button. When you activate 
any
\code{HELP} the system will show the appropriate
documentation page from which you can then access any part of the manual.

This manual is organized in eight parts to help you easily find the information you need. 
Each part presents one particular
program or module of the \COPRSDE{}. Because of the many interconnections
and cross references between these different parts, the manual is
structured to explain these cross references whenever
possible.

Different parts can be read in any order, although the order in which they
are presented is certainly the preferred one. It is suggested that you read a part describing a 
module before reading one describing the
same module's X11/Motif interface. For example, one should read the
\CPK{} part before the \XPK{} part.

\begin{description}

\item[Overview:]

\item[\CPK{}:] This part introduces the most important modules of the
\COPRSDE{}.  The \CPK{} is really the central program of the \COPRSDE{}. However,
it needs the other modules and programs to be used as a real application. It
executes the procedures produced with the \OPE{}. It is created and can be
interacted with, using the \OPRSS{}. It can be run under X11/Motif using the
\XOPRS{} interface. It can communicate with the external world and other \COPRS{}
kernels using the \MP{}. It represents the most important and biggest part of
the present manual.

\item[\OPRSS{}:] This part introduces the \OPRSS{} program which can create, kill
and allow the user to interact with \CPK{}s.

\item[\MP{}:] This part introduces the module which allows the \COPRS{} and
\XOPRS{} kernels to communicate with one another and with external programs.

\item[\XPK{}:] This is the X11/Motif companion of the \CPK{} module. In this
interface, you can graphically trace the procedures which are executing and
follow the various tasks the kernel is working on.

\item[\OPE{}:] This part describes the program which is used to create, edit
and modify OPs and procedures. It is a graphical editor based on X11/Motif.

\item[Using \COPRS{}:] This part describes how to use the \COPRSDE{}. A step by
step \OPE{} session is presented, as well as an example of how to run an \XPK{}
application. This part also goes through the process of explaining the various
choices you will need to make to set-up a \COPRS{} application, and gives some
example of such application.

\typeout{The description of the appendices talks about differences
  with SRI OPRS.  Should this be SRI PRS with no O?}

\item[Appendices:] The appendices describe various topics about the \COPRSDE{}:
installation (\pxref{How to Install the OPRS Development Environment}),
differences from SRI \OPRS{} (\pxref{Principal Differences from SRI OPRS}),
default OPs (\pxref{Default OPs}), the Grammar Used in the \COPRSDE{}
(\pxref{Grammar Used in the OPRS Development Environment}), etc., as well as a glossary, a bibliography
and various indices which point you back to the appropriate section of the
manual.

\end{description}

Some chapters or sections have arbitrarily been put in a particular
part, although they could equally well be presented in elsewhere.
For example, the OP syntax section (\pxref{OP Syntax and Semantic}) is
part of the \CPK{} part, although it could be in the \OPE{} part. In
any case, the various indices and cross references will always point
you to the proper section when necessary.

Comments and problems about this documentation should be reported to the
following electronic mail address: \t{oprs-manual@ingenia.fr}.

\typeout{Is this email address still accurate?}


%%% Local Variables: 
%%% mode: latex
%%% TeX-master: "oprs"
%%% End: 
