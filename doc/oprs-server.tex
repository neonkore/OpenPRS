%%%%%%%%%%%%%%%%%%%%%%%%%%%% -*- Mode: latex -*- %%%%%%%%%%%%%%%%%%%%%%%%%%
%%
%% Copyright (c) 1991-2003 Francois Felix Ingrand.
%% All rights reserved.
%%
%% Redistribution and use in source and binary forms, with or without
%% modification, are permitted provided that the following conditions
%% are met:
%%
%%    - Redistributions of source code must retain the above copyright
%%      notice, this list of conditions and the following disclaimer.
%%    - Redistributions in binary form must reproduce the above
%%      copyright notice, this list of conditions and the following
%%      disclaimer in the documentation and/or other materials provided
%%      with the distribution.
%%
%% THIS SOFTWARE IS PROVIDED BY THE COPYRIGHT HOLDERS AND CONTRIBUTORS
%% "AS IS" AND ANY EXPRESS OR IMPLIED WARRANTIES, INCLUDING, BUT NOT
%% LIMITED TO, THE IMPLIED WARRANTIES OF MERCHANTABILITY AND FITNESS
%% FOR A PARTICULAR PURPOSE ARE DISCLAIMED. IN NO EVENT SHALL THE
%% COPYRIGHT HOLDERS OR CONTRIBUTORS BE LIABLE FOR ANY DIRECT, INDIRECT,
%% INCIDENTAL, SPECIAL, EXEMPLARY, OR CONSEQUENTIAL DAMAGES (INCLUDING,
%% BUT NOT LIMITED TO, PROCUREMENT OF SUBSTITUTE GOODS OR SERVICES;
%% LOSS OF USE, DATA, OR PROFITS; OR BUSINESS INTERRUPTION) HOWEVER
%% CAUSED AND ON ANY THEORY OF LIABILITY, WHETHER IN CONTRACT, STRICT
%% LIABILITY, OR TORT (INCLUDING NEGLIGENCE OR OTHERWISE) ARISING IN
%% ANY WAY OUT OF THE USE OF THIS SOFTWARE, EVEN IF ADVISED OF THE
%% POSSIBILITY OF SUCH DAMAGE.
%%
%% $Id$
%%
%%%%%%%%%%%%%%%%%%%%%%%%%%%%%%%%%%%%%%%%%%%%%%%%%%%%%%%%%%%%%%%%%%%%%%%%%%%%%%%

\part{\OPRSS{}}
\node{OPRS-Server}
\cpindex{OPRS-Server}



\chapter*{Overview of the \OPRSS{}}
\node{Overview of the OPRS-Server}
\cpindex{Overview of the \OPRSS{}}

The \OPRSS{} is an important tool in the \COPRS{} development environment.
The philosophy behind \COPRS{} prevents the user from interacting synchronously
with a \CPK{}. However, one should be able to issue commands to a running
kernel without disturbing the \CPK{} main loop. This is where the \OPRSS{}
comes in.  The \OPRSS{} is a program which enables the user to interact
with a \CPK{} as much as possible. Moreover, the server allows the user to
create new kernels, kill them, and so on\dots{} As most \COPRSDE{} programs, the
\OPRSS{} will start the \MP{} if necessary.

The OPRS Server enables the user to communicate directly with \CPK{}s.
Indeed, \CPK{}s must be able to execute their procedures
without being disturbed synchronously by the user. That is the reason
why, the user can communicate with \CPK{}s through the OPRS Server.
This does not apply to \XPK{} with which the user can communicate
using the X11/Motif interface.

\chapter{How to Use the \OPRSS{}}
\node{How to Use the OPRS-Server}
\cpindex{How to Use the \OPRSS{}}
\pgindex{oprs-server command}



\section{Arguments of the \OPRSS{}}
\node{Arguments of the OPRS-Server}
\cpindex{Arguments of the \OPRSS{}}

Usage: \begin{verbatim}
oprs-server [-X] [-l upper|lower|none] [-i server-port-number]
           [-m message-passer-hostname]
           [-j message-passer-port-number]
           [-l upper|lower|none ]
\end{verbatim}

All the arguments are optional.

\begin{description}

\item[\code{-X}] to specify that all the \CPK{}s created by using the
\code{make} command in the \OPRSS{}, will be \XPK{}s.

\item[\code{-i}] to specify the port on which the \OPRSS{} is
listening to connections from \CPK{}s.

\item[\code{-m}] to specify the hostname on which the \MP{} is running (or will
be started).  If the \OPRSS{} cannot connect (even after starting it) to
this hostname on the specified  port, then the program exits with an error message.

\item[\code{-l upper|lower|none}] can be used to print and parse all the symbol
and id in upper case, lower case or in no particular case.All the kernels
created by this \OPRSS{} will inherit this property.

\item[\code{-j}] to specify the port on which the \MP{} is listening.

\end{description}

\section{\OPRSS{} Environment Variables}
\node{OPRS-Server Environment Variables}
\cpindex{OPRS-Server Environment Variables}

There are a number of environment variables which can be used to customize the
\OPRSS{} or to define default arguments. Arguments passed using the command line
have precedence on those acquired from environment variables.

\begin{description}

\cpindex{OPRS\_MP\_PORT}
\item[\code{OPRS\_MP\_PORT}] is used  to specify the port on which the \MP{} will
listen to connection. It is used by the \CPK{}, the \XPK{}, the \OPRSS{}
and the \MP{}. It is equivalent to the \code{-j} command line argument.\*
Example:
\begin{verbatim}
setenv OPRS_MP_PORT 3456
\end{verbatim}

\cpindex{OPRS\_MP\_HOST}
\item[\code{OPRS\_MP\_HOST}] is used  to specify the host on which the \MP{} will
listen to connection. It is used by the \CPK{}, the \XPK{}, the \MP{} and the
\OPRSS{}.  It is equivalent to the \code{-m} command line argument.\*
Example:
\begin{verbatim}
setenv OPRS_MP_HOST machine.site.domain
\end{verbatim}

\cpindex{OPRS\_SERVER\_PORT}
\item[\code{OPRS\_SERVER\_PORT}] is used  to specify the port on which the
\OPRSS{} will listen to connection. It is used by the \CPK{}, the \XPK{}
and the \OPRSS{}.  It is equivalent to the \code{-i} command line argument.\*
Example:
\begin{verbatim}
setenv OPRS_SERVER_PORT 3457
\end{verbatim}

\cpindex{OPRS\_ID\_CASE}
\item[\code{OPRS\_ID\_CASE}] is used to specify if the program should upper case,
lower case or should not change the case of the parsed Id. This is equivalent
to the \code{-l} option. The possible values
are \code{lower}, \code{upper} or \code{none}:\*
Example:
\begin{verbatim}
setenv OPRS_ID_CASE none
\end{verbatim}

\end{description}

\section{Commands of the \OPRSS{}}
\node{Commands of the OPRS-Server}
\cpindex{Commands of the \OPRSS{}}

Here is the list of the commands the \OPRSS{} recognizes.



\subsection{\OPRSS{} Commands to Handle \CPK{}}
\node{OPRS-Server Commands to Handle OPRS Kernel}
\cpindex{OPRS Server Commands to Handle \CPK{}}

\begin{itemize}

\pgindex{make (oprs-server)}
\item \code{make \var{name}}. To create a \CPK{} named \code{name}
(in a separate Unix process).

\pgindex{make\_x (oprs-server)}
\item \code{make-x \var{name}}. To create a \XPK{} named \code{name}
(in a separate Unix process).

\pgindex{kill (oprs-server)}
\item \code{kill \var{name}}. To kill the \CPK{} named \code{name}.
This can only work if the \CPK{} has been started with the
\code{make} command.

\pgindex{accept (oprs-server)}
\item \code{accept}. To accept the connection of a new \CPK{} (or a
\XPK{}). This is used whenever a kernel has been started from a Unix
shell (from a remote host for example) and is waiting for the \OPRSS{}
to accept its connection. There is a common mistake when one uses the
\code{accept} command. If you have started a \CPK{} and the
connection to the \OPRSS{} does not work after an \code{accept}, you
probably started this server with a port number for it or the \MP{} which is
not the default one. Remember to specify these numbers in the command line of
the \CPK{} you start.

\pgindex{connect (oprs-server)}
\item \code{connect \var{name}}. To connect the standard input to the \COPRS{}
agent named \code{name}. This puts the OPRS agent in \dfn{command}
mode.

\pgindex{disconnect (oprs)}
\item \code{disconnect}. To instruct the connected \COPRS{} to leave the
\code{stdin} and give it back to the \OPRSS{}. The OPRS client will
return in \dfn{run} mode. In fact, this command is not a \OPRSS{} command
but a \CPK{} command.

\pgindex{reset kernel (oprs-server)}
\item \code{reset kernel \var{name}}. To send a \code{reset kernel} command
to the \CPK{} named \var{name} (\pxref{OPRS Kernel Miscellaneous Commands}).

\pgindex{reset parser (oprs-server)}
\item \code{reset parser \var{name}}. To reset the parser of the
\CPK{} named \var{name}.

\end{itemize}

\subsection{\OPRSS{} Communication Commands}
\node{OPRS-Server Communication Commands}
\cpindex{OPRS Server Communication Commands}

\begin{itemize}

\pgindex{send (oprs-server)}
\item \code{send \var{name} \var{message}}. To send the \var{message} to the
\COPRS{} named \var{name}. \*
Example: \code{send foo (bar boo 3)}.

\pgindex{add (oprs-server)}
\item \code{add \var{name} \var{goal|fact}}. To send a \var{goal} or a
\var{fact} to the OPRS named \var{name}. This is how you can post new facts
or new goals in a \COPRS{} client. \*
Example for a fact: \code{add foo (bar boo 3)} or for a goal: \code{add foo
(! (print-factorial 3))}.

\pgindex{transmit (oprs-server)}
\item \code{transmit \var{name} \var{string}}. To send a \var{string} command
to a \COPRS{} client named \var{name}. This is how you can send commands to a
\COPRS{} client without connecting to it. If the command you want to send
contains double quotes (such as \code{include "data/foo.inc"}), then you
must backslash double quote like in: \code{transmit foo "include
\\"data/foo.inc\\" "}.

\pgindex{transmit\_all (oprs-server)}
\item \code{transmit\_all \var{string}}. To send a \var{string} command
to all the \CPK{} clients named connected to this \OPRSS{}. The syntax of the
string is similar to the one used in the \code{transmit} command.

\pgindex{broadcast (oprs-server)}
\item \code{broadcast \var{message}}. To send the \var{message} to
all the \COPRS{} connected to this \OPRSS{}'s \MP{}. \*
Example: \code{broadcast (bar boo 3)}.

\end{itemize}

\subsection{\OPRSS{} Miscellaneous Commands}
\node{OPRS-Server Miscellaneous Commands}
\cpindex{OPRS Server Miscellaneous Commands}

\begin{itemize}

\pgindex{q (oprs-server)}
\pgindex{quit (oprs-server)}
\pgindex{exit (oprs-server)}
\item \code{q|quit|exit|EOF}. To quit the \OPRSS{}. This also kills
all the \COPRS{} clients started by this \OPRSS{}.

\pgindex{include (oprs-server)}
\item \code{include \var{file\_name}}. To execute all the commands in
\var{file\_name}. The recommended extension for these files is \file{.inc}.
Include file can contain other include directives. Only two commands are
forbidden in include file: \code{connect} and \code{disconnect}
(\pxref{Include File Format}).

\pgindex{show version (oprs-server)}
\pgindex{show copyright (oprs-server)}
\item \code{show version|copyright}. To print the version or the copyright
notice.

\pgindex{help (oprs-server)}
\item \code{help|h|?}. To print some on-line help.

\end{itemize}


%%% Local Variables: 
%%% mode: latex
%%% TeX-master: "oprs"
%%% End: 
