%%%%%%%%%%%%%%%%%%%%%%%%%%%% -*- Mode: latex -*- %%%%%%%%%%%%%%%%%%%%%%%%%%
%%
%% Copyright (c) 1991-2010 Francois Felix Ingrand.
%% All rights reserved.
%%
%% Redistribution and use in source and binary forms, with or without
%% modification, are permitted provided that the following conditions
%% are met:
%%
%%    - Redistributions of source code must retain the above copyright
%%      notice, this list of conditions and the following disclaimer.
%%    - Redistributions in binary form must reproduce the above
%%      copyright notice, this list of conditions and the following
%%      disclaimer in the documentation and/or other materials provided
%%      with the distribution.
%%
%% THIS SOFTWARE IS PROVIDED BY THE COPYRIGHT HOLDERS AND CONTRIBUTORS
%% "AS IS" AND ANY EXPRESS OR IMPLIED WARRANTIES, INCLUDING, BUT NOT
%% LIMITED TO, THE IMPLIED WARRANTIES OF MERCHANTABILITY AND FITNESS
%% FOR A PARTICULAR PURPOSE ARE DISCLAIMED. IN NO EVENT SHALL THE
%% COPYRIGHT HOLDERS OR CONTRIBUTORS BE LIABLE FOR ANY DIRECT, INDIRECT,
%% INCIDENTAL, SPECIAL, EXEMPLARY, OR CONSEQUENTIAL DAMAGES (INCLUDING,
%% BUT NOT LIMITED TO, PROCUREMENT OF SUBSTITUTE GOODS OR SERVICES;
%% LOSS OF USE, DATA, OR PROFITS; OR BUSINESS INTERRUPTION) HOWEVER
%% CAUSED AND ON ANY THEORY OF LIABILITY, WHETHER IN CONTRACT, STRICT
%% LIABILITY, OR TORT (INCLUDING NEGLIGENCE OR OTHERWISE) ARISING IN
%% ANY WAY OUT OF THE USE OF THIS SOFTWARE, EVEN IF ADVISED OF THE
%% POSSIBILITY OF SUCH DAMAGE.
%%
%% $Id$
%%
%%%%%%%%%%%%%%%%%%%%%%%%%%%%%%%%%%%%%%%%%%%%%%%%%%%%%%%%%%%%%%%%%%%%%%%%%%%%%%%

\part{\MPA{}}
\node{Message Passer}



\chapter*{Overview of the \MPA{}}
\node{Overview of the Message Passer}
\cpindex{Overview of the \MPA{}}
\cpindex{Introduction to the \MPA{}}


The Message Passer is the program which allows an application to communicate
with \CPK{}s and \XPK{}s. The \MPA{} has one and only one function: it passes
messages between various programs which have been registered. Most of the time,
it is started by the \OPRSS{}, a \XPK{} or a \CPK{}, so you do not have to call
it directly.  However, you can if you want, start it on any host, with any port
on which to listen. In this case, you have to call it with the port number you
want it to listen. \xref{Argument of the Message Passer}, for details. There is
a companion program to the \MPA{}: \code{kill-mp} which can be used to kill the
\MPA{}. Moreover, if the \MPA{} has no client registered for more than five
hours, it will exit (to make sure your machine is not loaded by unused
\MPA{} processes). 

The main characteristic of the \MPA{} are:

\begin{itemize}

\item Communication using TCP/IP, the most popular communication media
and protocol on the  Unix operating system.

\item Communication on heterogeneous network. One can have a \XPK{}
on a Sparc Station communicating with a \CPK{} on a DEC Station,
while the application run on a VAX.

\item Various protocols available between the simulators, the
applications and the Message Passer.

\item Easy to use from your application, using registration and
communication functions provided in a library .

\end{itemize}

The \MPA{} is identified by a host name and a port number. In other words,
one can potentially run as many \MPA{} as desired on a network of
heterogeneous machines. Usually, one will run one \MPA{} for one \COPRS{}
application. This \MPA{} serves as the central server for messages and
information passing between the different programs involved in a
specific application.

By default the \MPA{} listens on the port 3300, however, there are means to get
it to listen onto another port. This new port can currently be specified when
you start the \OPRSS{} using the appropriate argument (\pxref{Arguments of the OPRS-Server}, for details), or when you start the \MPA{} on its own.

\chapter{How to Use the Message Passer}
\node{How to Use the Message Passer}
\cpindex{How to Use the Message Passer}

The various programs of the \COPRS{} development environment connect to the
\MPA{} without any intervention of the user. However, if you write your own
module and want it to communicate with the \MPA{}, then you need to
register it and follow a particular protocol.

The registration mainly consists in connecting to a public TCP/IP Internet
socket, and in sending the desired protocol and its name. There are two
possible protocols for connection to the message passer. After the
registration, the client (which in most cases is a \CPK{}) is supposed to check
from time to time its \code{mp-socket} by selecting it (in C) or listening to
it (in Lisp).  If something is present, a message is available and should be
read promptly.  Similarly, when a message has to be sent to another \COPRS{}
kernel, or to an external module which has registered to the \MPA{}, it is just
a matter of writing two strings on the socket \code{mp-socket}: one for the
name of the recipient and one for the message itself. These strings are sent in
a particular format, and the user should not attempt to send them directly but
instead use the library functions provided for this very purpose.

Note that each module needs to connect to the \MPA{} with a unique name.

The \MPA{} can be killed or shutdown using the \code{kill-mp} program. 



\section{Argument of the \MPA{}}
\node{Argument of the Message Passer}
\cpindex{Argument of the \MPA{}}


Although it is seldom started from the Unix shell, one can start the \MPA{} on
its own. In this case, you have to specify the proper arguments in the
\OPRSS{} command line (\pxref{Arguments of the OPRS-Server}) and the \OPRS{}
command line (\pxref{Arguments to the oprs Command}) .

Usage: \begin{verbatim}
mp-oprs [-j message-passer-port-number]
       [-l filename] [-x] [-v] 
\end{verbatim}


All the arguments are optional.

\begin{description}

\item[\code{-j}] to specify the port on which the \MPA{} is listening.

\item[\code{-l filename}] can be used to log in the file \file{filename} all
the messages passed by the \MPA{}.

\item[\code{-v}] to specify a verbose mode for the \MPA{}. In verbose mode, all
messages passed by the \MPA{} are traced.

\item[\code{-x}] to specify that any new registration made to the \MPA{} with an
already registered name lead to the deconnection of the old client. The default
behavior is to refuse the connection to client with name already used.

\end{description}

\section{\MPA{} Environment Variables}
\node{Message Passer Environment Variables}
\cpindex{Message Passer Kernel Environment Variables}

There is one environment variable which can be used to customize the \MPA{}.
However, the argument passed using the command line has precedence on the one
acquired from the environment variables.

\begin{description}

\cpindex{OPRS\_MP\_PORT}
\item[\code{OPRS\_MP\_PORT}] is used  to specify the port on which the \MPA{} will
listen to connection. It is used by the \CPK{}, the \XPK{}, the \OPRSS{}
and the \MPA{}.\*
Example:
\begin{verbatim}
setenv OPRS_MP_PORT 3456
\end{verbatim}

\end{description}

\section{Argument of the \MPK{}}
\node{Argument of the Message Passer Killer}
\cpindex{Argument of the \MPK{}}

One can use the \code{kill-mp} program to kill a \MPA{}. Note that this program
can kill any \MPA{} you started on any host. So this command should be used with
extreme caution. The \MPA{} will check that the \MPK{} program has been started
by the same user than the one who started it (the \MPA{}). Moreover, the super
user (the user root) can kill any \MPA{}.  

Usage: \begin{verbatim}
kill-mp [-m message-passer-hostname]
        [-j message-passer-port-number]       

\end{verbatim}

All the arguments are optional.

\begin{description}

\item[\code{-m}] to specify the hostname on which the \MPA{} is running (and will
be killed).

\item[\code{-j}] to specify the port on which the \MPA{} is listening (and will
connected to to be killed).

\end{description}

\section{\MPK{} Environment Variables}
\node{Message Passer Killer Environment Variables}
\cpindex{Message Passer Killer Kernel Environment Variables}

There are two environment variable which can be used to customize the
\MPK{}. However, the  argument passed using the command line
have precedence on the one acquired from the environment variables.

\begin{description}

\cpindex{OPRS\_MP\_PORT}
\item[\code{OPRS\_MP\_PORT}] is used  to specify the port on which the \MPK{}
connect to kill the \MPA{}. It is equivalent to the \code{-j} command line
argument.\* 
Example:
\begin{verbatim}
setenv OPRS_MP_PORT 3456
\end{verbatim}

\cpindex{OPRS\_MP\_HOST}
\item[\code{OPRS\_MP\_HOST}] is used  to specify the host on which the \MPK{} will
connect to kill the \MPA{}. It is equivalent to the \code{-m} command line argument.\* 
Example:
\begin{verbatim}
setenv OPRS_MP_HOST machine.site.domain
\end{verbatim}

\end{description}

\section{How to Connect to the \MPA{} from \OPRSS{} and \CPK{}}
\node{How to Connect to the Message Passer from OPRS Server and OPRS}
\cpindex{How to Connect to the \MPA{} from \OPRSS{} and \CPK{}}

There is nothing particular to do for the \OPRSS{} or a \CPK{} to use the
\MPA{}. For all these programs, the registration to the \MPA{} is mandatory and
automatic. Upon starting, they register to the \MPA{}, and they regularly check
if something is present for them on the appropriate socket.

\section{How to Connect to the \MPA{} from an External Module}
\node{How to Connect from an External Module}
\cpindex{How to Connect to the \MPA{} from an External Module}

Any external module or program can register to the \MPA{}. It is even the only
way for an arbitrary program to communicate with \CPK{}s. By default, the \MPA{}
always reports on \code{stderr} the registration of a new client. There are two
different protocols: the \code{MESSAGES\_PT} protocol and the \code{STRINGS\_PT}
protocol. When you establish a connection, you need to specify the protocol
connection. Note that if you use the registration function without protocol
specification, the rule to define the protocol is the following. The
\code{MESSAGES\_PT} protocol is the default one, to use the \code{STRINGS\_PT}
protocol, you need to suffix the name of your module with a \samp{/}. So if
your module is called \code{foo}, you should send the string \code{foo/} on the
socket for registration. However, the module will be known under the \code{foo}
name. This mechanism remains for upward compatibility. In any case, you should
try to use the registration function with the protocol specification.

\code{mp\_port, external\_register\_to\_the\_mp\_host, external\_register\_to\_the\_mp}
are not supported anymore.

The registration functions and variables are provided in the
\file{libmp.a} library for this purpose. Their prototype is defined in
\file{mp-pub.h}.

The following protocol types are currently supported: 
\begin{verbatim}
typedef enum {MESSAGES_PT, STRINGS_PT} Protocol_Type;
\end{verbatim}

\begin{typevr}{MP Library Variable}{int}{mp\_socket}
is the \MPA{} socket on which a program can send messages to and receive
messages from the \MPA{}.
\end{typevr}


\begin{typevr}{MP Library Variable}{char *}{mp\_name}
is the name of the client as set by the \MPA{}. It is usually the name you gave
as argument.
\end{typevr}

\begin{typefn}{MP Library Function}{int}{external\_register\_to\_the\_mp\_host\_prot}
        {(char *\var{name}, char *\var{host\_name}, int \var{port},
Protocol\_Type \var{prot})} registers the calling program to the \MPA{} under the
name \var{name}, on the host specified in \var{host\_name}, on the port
\var{port}, with the protocol specified in \var{port}.
\end{typefn}

\begin{typefn}{MP Library Function}{int}{external\_register\_to\_the\_mp\_prot}
        {(char *\var{name},  int \var{port}, Protocol\_Type \var{prot})}
is similar to the previous function except that the host on which the
\MPA{} is expected to run is the same as the one on which the
program runs.
\end{typefn}

These two functions set and return the value of \code{mp\_socket}. If this value
is -1, then the connection attempt failed.

\section{Messages Format}
\node{Messages Format}
\cpindex{Messages Format}

The format of data between sender and recipient using the \MPA{} depends on the
established protocol. It is up to each module to parse and interpret them as
they arrive. From the sender to the \MPA{}, the format is always two strings
(one for the recipient and one for the message). However, from the \MPA{} to a
module, it depends. In \code{MESSAGES\_PT} mode, the string \code{"receive
<name-sender> <message>"} is sent as is. In \code{STRINGS\_PT} mode, the name and
the message are sent as two separate strings, each string being sent as an int
(representing its size) and a byte stream for the string itself.

For example, in \code{MESSAGES\_PT} mode, if the \CPK{} \code{FOO} wants to send
the message \code{(position valve1 closed)} to the \CPK{} \code{BAR}, then it
calls the function send\_message. The \MPA{} sees two strings coming on
\code{FOO}'s socket: \* \code{"BAR"} the recipient, and \code{"(POSITION VALVE1
CLOSED)"} the printed representation of the message. After processing it, the
\MPA{} sends one string to \code{BAR}: \* \code{"receive FOO (POSITION VALVE1
CLOSED)"} which is in fact interpreted as a command by the \code{BAR} kernel
that the message \code{(POSITION VALVE1 CLOSED)} has been received from the
kernel \code{FOO}.

In \code{STRINGS\_PT} mode, the name and the message are sent as two separate
strings, each string being sent as an int for its size and a byte stream for
the string.

Here are the functions provided to read, write, and send message. Their
prototype is defined in \file{mp-pub.h}.

\begin{typefn}{MP Library Function}{PString}{read\_string\_from\_socket}
        {(int \var{socket}, int *\var{size})}
is the function, in the \file{libmp.a} library, which can be used to read a
string from the mp-socket in \code{STRINGS\_PT} protocol. This function needs to be
called twice, one time for the name of the sender, and one time for the text of
the message. The \var{size} parameter is a return parameter to tell you how big
the string is.  Do not forget to free the result with the \code{free} function.
\end{typefn}

\begin{typefn}{MP Library Function}{void}{send\_message\_string}
        {(PString \var{message}, PString \var{rec})}
is the function, in the \file{libmp.a} library, which can also be used to
send a message \var{message} to the recipient \var{rec}. This function
\code{send\_message\_string} is preferred to the former function
\code{write\_string\_to\_socket}.
\end{typefn}

\begin{typefn}{MP Library Function}{void}{multicast\_message\_string}
        {(PString \var{message}, unsigned int \var{nb\_recs}, PString \var{*recs})}
is the function, in the \file{libmp.a} library, which can be used to
send a message \var{message} to  a list of \var{nb\_recs} recipient which names
are in the array of PString \var{recs}. 
\end{typefn}

\begin{typefn}{MP Library Function}{void}{broadcast\_message\_string}
        {(PString \var{message})}
is the function, in the \file{libmp.a} library, which is used to send a message
\var{message} to  all the connected agents, except the sender.  
\end{typefn}

\section{Example of C Code to Connect to the \MPA{}}
\node{Example of C Code to Connect to the Message Passer}
\cpindex{Example of C Code to Connect to the \MPA{}}

This code comes from the Truck Loading Demo (\xref{Truck Loading Example}), and
can be found in the file \file{demo/truck-demo/src/oprs-interface.c}.

\begin{verbatim}
void demo_init_arg(int argc, char **argv)
{
     int c, getoptflg = 0;
     int mpname_flg = 0, mpnumber_flg = 0, demoname_flg = 0;

     struct hostent *check_hostname;
     int mp_port;
     extern int optind;
     extern char *optarg;
     int maxlength = MAX_HOST_NAME * sizeof(char);

     while ((c = getopt(argc, argv, "m:j:n:h")) != EOF) {
          switch (c)
          {
          case 'm':
               mpname_flg++;
               mp_host_name = optarg;
               break;
          case 'j':
               mpnumber_flg++;
               if (!sscanf (optarg, "%d", &mp_port ))
                    getoptflg++;
               break;
          case 'n':
               demoname_flg++;
               demo_name = optarg;
               break;
          case 'h':
          default:
               getoptflg++;

          }
     }
     if (getoptflg) {
          fprintf(stderr, DEMO_ARG_MESSAGE );
          exit(1);
     }

     if (mpname_flg){
          if ((check_hostname = gethostbyname (mp_host_name)) == NULL){
               fprintf(stderr, "Invalid mp host name \n");
               exit (1);
          }
     } else {
          mp_host_name = (char *)malloc (maxlength);
          if (gethostname(mp_host_name, MAX_HOST_NAME) != 0) {
               fprintf(stderr, "Error in gethostname \n");
               exit(1);
          }
     }
     if (!mpnumber_flg)
          mp_port = MP_PORT;

     if ( !demoname_flg ){
          demo_name = default_demo_name;
          connect_name = default_connect_name;
     } else {
          int i, length = strlen (demo_name);

          if (demo_name[length-1] != '/') { /* The name doesn't end with a
                                               '\', add it to get the
                                               right message format . */

               connect_name = (char *)malloc (length +2) ;
               for (i = 0; i< length ; i++){
                    if (islower(demo_name [i]))
                         connect_name[i] = toupper (demo_name[i]);
                    else
                         connect_name[i] = demo_name[i];
               }
               connect_name[length] = '/';
               connect_name[length + 1] = '\0';

          }
     }
}

void send_message_to_oprs (char *message)
{
     char trace_message[BUF_SIZE];

     if (!demo->connected) {
          demo_error ("send_message: You are not connected ");
          return;
     }
     send_message_string(message, OPRS_NAME);

     sprintf (trace_message, "Send: %s\n", message);
     oprs_message(trace_message);

}

void get_oprs_message (XtPointer client_data, int *fid, XtInputId id)
{
     char trace_message[BUF_SIZE];
     int length;
     char *sender;
     char *message;

     sender = read_string_from_socket(*fid, &length);
     message = read_string_from_socket(*fid, &length);

     if ( decode_message (message, sender) != 0){
          sprintf (trace_message, "Received %s from %s \n", message, sender);
          demo_error (trace_message);
     }

     free(sender);
     free(message);
}

void connect_to_mp ()
{
     if ((mp_socket = external_register_to_the_mp_host_pfrot(connect_name,
mp_host_name, mp_port, STRINGS_PT)) == -1) {
          demo->connected = FALSE;
          demo_warning ("Unable to register to the Message Passer");
     } else {
          demo_message ("You are connected to the Message Passer\n");
          demo->connected = TRUE;
          XtAppAddInput (app_context,
                         mp_socket,
                         XtInputReadMask,
                         get_oprs_message,                 /* the read function */
                         NULL);
     }
}

\end{verbatim}

\section{Example of Lisp Code to Connect to the \MPA{}}
\node{Example of Lisp Code to Connect to the Message Passer}
\cpindex{Example of Lisp Code to Connect to the \MPA{}}

Here are the foreign function definitions to register a lisp program to
the \MPA{}, and to send messages to the \MPA{}:

\begin{verbatim}
(in-package "OPRS" :use '("LISP" "LUCID-COMMON-LISP"))

;;; Loading

(defun i-oprs-load ()
  (load-foreign-libraries
   nil
   (list "/usr/local/oprs/lib/libmp.a"
         "-lm"
         "-lc")))

;;; functions  base

(def-foreign-function
  (i-external-register-to-the-mp
        (:name "_external_register_to_the_mp")
        (:language :c)
        (:return-type :signed-32bit))
  (name :string))

(def-foreign-function
  (i-external-register-to-the-mp-host
        (:name "_external_register_to_the_mp_host")
        (:language :c)
        (:return-type :signed-32bit))
  (name :string)
  (host :string))

(def-foreign-function
  (i-send-message-string
        (:name "_send_message_string")
        (:language :c)
        (:return-type :null))
  (msg :string)
  (target :string))

(i-oprs-load)
\end{verbatim}

Here are some lisp functions to be used to communicate with the
\MPA{}. Note that as \COPRS{} uses a lisp like syntax, we can use read to
read the message coming from the \CPK s.

\begin{verbatim}
(in-package "OPRS" :use '("LISP"))

(defvar *mp-stream* nil)

(defun valid-sd (sd)
  (and (integerp sd) (> sd 0)))

(defun already-registered ()
  (streamp *mp-stream*))

(defun oprs-register-to-mp (name &optional (machine (machine-instance)))
  (check-type name string)
  (check-type machine string)
  (when (already-registered)
    (warn "Connection to the Message Passer already established.")
    (return-from oprs-register-to-mp nil))
  (let ((sd (i-external-register-to-the-mp-host name machine)))
    (cond ((valid-sd sd)
           (setf *mp-stream* (make-lisp-stream :input-handle sd
                                                :output-handle sd
                                                 :auto-force t))
           t)
          (t (warn "Failed to connect to the Message Passer."))
          nil)))

(defun oprs-close ()
  (break-if-not-registered)
  (close *mp-stream*)
  (setf *mp-stream* nil))

;;;---
;;; writing
;;;---

(defun oprs-write (&key target msg)
  (check-type target string)
  (break-if-not-registered)
  (i-send-message-string (format nil "~S" msg) target))

;;;---
;;; reading
;;;---

(defun oprs-listen ()
  (break-if-not-registered)
  (loop
   (if (and (listen *mp-stream*)
            (member (peek-char nil *mp-stream*)
                    '(#\space
                      #\newline
                      #\linefeed
                      #\return
                      #\page
                      #\backspace
                      #\tab
                      #\)
                      )))
       (read-char *mp-stream*)
       (return)))
  (listen *mp-stream*))

(defstruct oprs-msg
  sender
  contents
)

(defun oprs-read ()
  (break-if-not-registered)
  (if (oprs-listen)
      (progn
        (read *mp-stream*) ; This is to remove the "receive" keyword
        (make-oprs-msg :sender (read *mp-stream*)
                       :contents (read *mp-stream*)))
      :NO-MSG))

(defun break-if-not-registered ()
  (unless (already-registered)
    (break "No connection with the Message Passer.")))

\end{verbatim}

Here are the entries you may want to export out of the \code{OPRS}
package.

\begin{verbatim}
(in-package "OPRS" :use '("LISP" "LUCID-COMMON-LISP"))

(defvar *oprs-exports*)

(eval-when (load eval compile)
  (setf *oprs-exports*
        '(
          oprs-register-to-mp
          oprs-close

          oprs-write
          oprs-listen
          oprs-read

          oprs-msg-sender
          oprs-msg-contents
          )))

(export *oprs-exports*)
\end{verbatim}

\section{Errors Reported by the \MPA{}}
\node{Errors Reported by the Message Passer}
\cpindex{Errors Reported by the \MPA{}}

A certain number of errors can be reported by the \MPA{}. In
general, the \MPA{} is very verbose and notifies the user of
any unexpected event or situation. Whenever it is possible, the reported
message  indicates the host and the port number on which this \MPA{} is
running. The most common problems are: 

\begin{description}

\item[\code{"Disconnecting the client: \%s from the message passer."}]
This usually happens when a client died and the communicating socket
has been closed. When the \MPA{} realizes this, it prints this message.

\item[\code{"Registering the client: \%s with protocol: \%s."}] is printed upon
    successful connection to the \MPA{}.

\item[\code{"logging output in file `\%s'."}] is printed whenever the \MPA{} log
    its output to a file.

\item[\code{"nobody registered for more than \%d seconds, mp-oprs (\%d): exit."}]
    is printed when the \MPA{} exits when there is no connection  and no new
    connection have been made in a certain amount of time.

\item[\code{"already has a client named: \%s. Denying registration."}]
This problem is reported if a new client tries to register
with a name already used by somebody else.

\item[\code{"EOF in get\_and\_send\_message (recipient) from \%s."}]
This error is reported when the \MPA{} gets an End of File while
reading the name of the recipient on a socket. Most often it appears
when a client dies.

\item[\code{"EOF in get\_and\_send\_message, (message) from \%s."}]
This error is reported when the \MPA{} gets an End of File while
reading the message part of a pair recipient-message on a socket.

\item[\code{"A message has been sent to \%s, but no such agent exists."}]
This error is reported if a message is sent to a an unknown client.

\item[\code{"unknown message type in get\_and\_send\_message from \%s."}] This
    error occurs whenever a message with an unknown type is received by the
    message passer.

\item[\code{"Disconnecting the client: \%s from the message passer."}] This
    message is printed when a client is disconnected by the \MPA{}.

\item[\code{"kill request, checking identity."}] A kill request has been
  received by the \MPA{} which checks if it has been sent by an authorized client.

\item[\code{"denying kill-mp, you are not the user who started this message
    passer"}] is printed when a the client which sent the kill request is not
  authorized.

\item[\code{"shutting down the message passer socket."}] is printed when the
  \MPA{} exit after a kill request has been sent.

\item[\code{"A message could not be delivered to \%s."}]
This error is reported if a message cannot be properly delivered to
its recipient for some unknown reason.

\end{description}


%%% Local Variables: 
%%% mode: latex
%%% TeX-master: "oprs"
%%% End: 
